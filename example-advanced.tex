%%%
% This advanced example of a B+ tree is supposed to help creating/modifying B+
% trees using the btreevis package. It covers all possibilities of extending and
% shrinking a B+ tree. Hence, it is a good way to learn and understand what are
% the necessary steps to create and modify B+ trees, i.e. which macros/styles
% have to be used/modified.
%
% All sections and subsections provide a header describing what has to be done 
% and why. Each (sub-)section contains exactly a single tikzpicture containing
% inline comments to describe single parts of the header description.
%
% The (sub-)sections are described in terms of the (sub-)section before. So, if
% a change is described, it is a change with respect to the (sub-)section before.
%%%

\documentclass{article}

\usepackage{hyperref}

% Use btreevis package.
\usepackage{btreevis}

\begin{document}
  \tableofcontents
  \clearpage

  %%%
  % Create an intially empty B+ tree with one node containing 3 key cells and
  % 4 pointer cells.
  %%%
  \section{Initial B$^+$ tree}  
  \begin{center}
  \begin{tikzpicture}[
    BTreeDefault,             % Use default B+ tree style.
    scale = 1                 % No scaling necessary.
  ] 
    % Set number of pointers per node (this should be done always upfront).
    \SetNoOfPoinersPerNode{4}
  
    % Generate (empty) content of (tree) level 0 (root level).
    \CreateBTreeNode{0}{1}{{}}{\levelZeroNodeOne}

    % Generate B+ tree with a single node (no content).
    \node[BTreeNodeDefault] {\levelZeroNodeOne};

    % As there is just a single node by now, no nodes need to be connected.
  \end{tikzpicture}
  \end{center}

  %%%
  % This section describes all possibilities of extending a B+ tree:
  %   - Add values to an existing node when a node is not yet full.
  %   - Add values to an existing node when a node is already full.
  %   - Add a new level to the B+ tree and connect the new nodes/leaves.
  %   - Add a new node to an existing level and connect the new node.
  %%%
  \section{Extending the B$^+$ tree}

  %%%
  % Values 3 and 9 are added to the existing (empty) node.
  % Hence, the node is neither full before nor after the inserts.
  % 
  % This is probably the easiest thing to do: we simply add the values 3 and 9
  % to the comma-separated list of values (which is empty by now) as third
  % argument of the \CreateBTreeNode macro. The third values is treated as empty
  % and thus the cell will also have no content in it.
  %%% 
  \subsection{Insert 3 and 9}
  \begin{center}
  \begin{tikzpicture}[
    BTreeDefault,             % Use default B+ tree style.
    scale = 1                 % No scaling necessary.
  ] 
    % Set number of pointers per node (this should be done always upfront).
    \SetNoOfPoinersPerNode{4}
  
    % Generate content of (tree) level 0 (root level).
    % Here we add 3 and 9 to the comma-separated list, i.e. {3, 9}.
    \CreateBTreeNode{0}{1}{{3, 9}}{\levelZeroNodeOne}

    % Generate B+ tree with a single node.
    % No change here because we still have only a single node.
    \node[BTreeNodeDefault] {\levelZeroNodeOne};

    % As there is still just a single node by now, no nodes need to be connected.
  \end{tikzpicture}
  \end{center}

  %%%
  % Value 10 is added to the existing node containing two values by now.
  % The node is full afterwards.
  %
  % As the workflow is the same, this should be as easy to understand as the
  % last step: we again add the value 10 to the comma-separated list of values.
  %%%
  \subsection{Insert 10}
  \begin{center}
  \begin{tikzpicture}[
    BTreeDefault,             % Use default B+ tree style.
    scale = 1                 % No scaling necessary.
  ] 
    % Set number of pointers per node (this should be done always upfront).
    \SetNoOfPoinersPerNode{4}
  
    % Generate content of (tree) level 0 (root level).
    % Here we add 10 to the comma-separated list, i.e. {3, 9, 10}.
    \CreateBTreeNode{0}{1}{{3, 9, 10}}{\levelZeroNodeOne}

    % Generate B+ tree with a single node.
    % No change here because we still have only a single node.
    \node[BTreeNodeDefault] {\levelZeroNodeOne};

    % As there is still just a single node by now, no nodes need to be connected.
  \end{tikzpicture}
  \end{center}

  %%%
  % Values 1 and 2 are added to the existing node which is already full. Because
  % the node is already full we need to split the node into two and add another
  % level to the tree.
  %
  % Several things have to be done:
  %   - Add two new calls of the \CreateBTreeNode macro to generate a total of 
  %     three node contents: one root node and its two child nodes.
  %   - When generating the tree, we need two new 'child' clauses to utilize the
  %     tree environment of TikZ to position the nodes properly. We use the newly
  %     generated contents as node contents.
  %   - As we have two tree levels, the \ConnectBTreeNodes and the
  %     \ConnectBTreeLeaves macros have to be used to connect the nodes in a
  %     proper way. Hence, we introduce one call for each of these two macros.
  %   - Because of the two tree levels it is also recommended to add a style for
  %     the new level of the tree.
  %%%
  \subsection{Insert 1 and 2}
  \begin{center}
  \begin{tikzpicture}[
    BTreeDefault,             % Use default B+ tree style.
    scale = 1,                % No scaling necessary.
    level 1/.style = {        % Level 1 tree style recommended.
      sibling distance = 8em,
      level distance = 6em
    }
  ] 
    % Set number of pointers per node (this should be done always upfront).
    \SetNoOfPoinersPerNode{4}
  
    % Generate content of (tree) level 0 (root level).
    % Here we just have a single value (9) in the comma-separated list because
    % we only have two subsequent nodes.
    \CreateBTreeNode{0}{1}{{9}}{\levelZeroNodeOne}

    % Generate content of (tree) level 1.
    % We add two new calls of the \CreateBTreeNode macro to generate the contents
    % of the subsequent nodes. One node contains three values, i.e. the comma-
    % separated list is {1, 2, 3}, and the second node contains two values, i.e.
    % {9, 10}.
    % We also have to provide the correct level number (1) as first argument of
    % both calls as well as the correct node number for each call as second
    % argument, i.e. 1 and 2 respectively.
    % As last argument we need to specify two new variables in which the
    % respective node content is stored.
    \CreateBTreeNode{1}{1}{{1, 2, 3}}{\levelOneNodeOne}
    \CreateBTreeNode{1}{2}{{9, 10}}{\levelOneNodeTwo}

    % Generate B+ tree with one root and two child nodes.
    % Here we add two 'child' clauses generating two new tree nodes. Because
    % these nodes are on the subsequent tree level (level number 1), we use the
    % two contents generated for this level as node contents.
    \node[BTreeNodeDefault] {\levelZeroNodeOne}
    child { node[BTreeNodeDefault] {\levelOneNodeOne} }
    child { node[BTreeNodeDefault] {\levelOneNodeTwo} };

    % Because the tree now consists of three nodes and two levels, we need to
    % use the \ConnectBTreeNodes macro to connect all the nodes vertically.
    % We connect from level 0 (first argument) node 1 (second argument) to
    % level 1 (third argument). On level 1 we have 2 nodes to connect (fourth
    % argument) and the node to start with on level 1 has number 1 (fifth
    % argument).
    \ConnectBTreeNodes{0}{1}{1}{2}{1}

    % Because the tree now consists of more than one leaf node, we need to
    % use the \ConnectBTreeLeaves macro to connect all the leaves horizontally.
    % We connect all leaf nodes of level 1 (first argument) and there are 2 leaf
    % nodes to connect. Remember: the \ConnectBTreeLeaves macro always starts
    % connecting from node number 1 of the supplied level, so no node to start
    % with has to be defined.
    \ConnectBTreeLeaves{1}{2}
  \end{tikzpicture}
  \end{center}

  %%%
  % Values 6 and 4 are added to the first node of the B+ tree. Hence, this node
  % is overfilled and has to be split into two nodes. So another node must be
  % added to the existing leaf level of the tree.
  %
  % To accomplish this, the following steps have to be done:
  %   - The root node has to be modified. Because we add another node on the
  %     subsequent level, the root node needs to hold another value, i.e. 3.
  %   - Obviously, one new call to the \CreateBTreeNode is necessary to generate
  %     the content of the new leaf node. Thus, also a new variable to hold this
  %     content has to be declared. Furthermore, the node is located between the
  %     two existing leaf nodes. Because of the consecutive numbering of the
  %     nodes, we also have to provide the correct node numbers for the new and
  %     the last leaf node!
  %   - A new child has to be added to the tree containing the newly generated
  %     node content.
  %   - The \ConnectBTreeNodes and \ConnectBTreeLeaves calls have to adjusted
  %     according to the new number of leaf nodes.
  %%%
  \subsection{Insert 6 and 4}
  \begin{center}
  \begin{tikzpicture}[
    BTreeDefault,             % Use default B+ tree style.
    scale = 1,                % No scaling necessary.
    level 1/.style = {        % Level 1 tree style recommended.
      sibling distance = 8em,
      level distance = 6em
    }
  ] 
    % Set number of pointers per node (this should be done always upfront).
    \SetNoOfPoinersPerNode{4}
  
    % Generate content of (tree) level 0 (root level).
    % Here we have to add a value (3) to the comma-separated list.
    \CreateBTreeNode{0}{1}{{3, 9}}{\levelZeroNodeOne}

    % Generate content of (tree) level 1.
    % We add one new call of the \CreateBTreeNode macro to generate the content
    % one additional leaf node. Because of the splitting, the new node is located
    % between the two already existing leaves. Hence, we also have to adjust the
    % node numbers of this level: the previously node 2 now has node number 3 and
    % the new leaf node gets node number 2. Also a new variable is introduced and
    % the variables are used according to the node numbering.
    % Summary:
    %   - Node 1 holds values 1 and 2.
    %   - Node 2 holds values 3, 4 and 6.
    %   - Node 3 holds values 9 and 10.
    \CreateBTreeNode{1}{1}{{1, 2}}{\levelOneNodeOne}
    \CreateBTreeNode{1}{2}{{3, 4, 6}}{\levelOneNodeTwo}
    \CreateBTreeNode{1}{3}{{9, 10}}{\levelOneNodeThree}

    % Generate B+ tree with one root and three child nodes.
    % Here we add another 'child' clauses generating one new tree node. As before
    % we use the previously generated node contents.
    \node[BTreeNodeDefault] {\levelZeroNodeOne}
    child { node[BTreeNodeDefault] {\levelOneNodeOne} }
    child { node[BTreeNodeDefault] {\levelOneNodeTwo} }
    child { node[BTreeNodeDefault] {\levelOneNodeThree} };

    % Because the root node now has three child nodes, we need to modify the
    % the \ConnectBTreeNodes call accordingly.
    % We connect from level 0 (first argument) node 1 (second argument) to
    % level 1 (third argument). On level 1 we now have 3 nodes to connect (fourth
    % argument) and the node to start with on level 1 has number 1 (fifth
    % argument).
    \ConnectBTreeNodes{0}{1}{1}{3}{1}

    % Because the tree now consists of one more leaf node, we also need to
    % use the \ConnectBTreeLeaves macro accordingly.
    % We connect all leaf nodes of level 1 (first argument) and now there are 3
    % of those to connect.
    \ConnectBTreeLeaves{1}{3}
  \end{tikzpicture}
  \end{center}

  %%%
  % Values 5 and 7 are added to the second node of the B+ tree. Hence, this node
  % is overfilled and has to be split into two nodes. So another node must be
  % added to the existing leaf level of the tree.
  %
  % To accomplish this, the following steps have to be done:
  %   - The root node has to be modified. Because we add another node on the
  %     subsequent level, the root node needs to hold another value, i.e. 5.
  %   - Once again, a new call to the \CreateBTreeNode is necessary to generate
  %     the content of the new leaf node. Thus, also a new variable to hold this
  %     content has to be declared. Furthermore, the node is located between the
  %     second and the third existing leaf nodes. Because of the consecutive
  %     numbering of the nodes, we also have to adjust the node numbers for the
  %     new and all following leaf nodes.
  %   - A new child has to be added to the tree containing the newly generated
  %     node content.
  %   - The \ConnectBTreeNodes and \ConnectBTreeLeaves calls have to adjusted
  %     according to the new number of leaf nodes.
  %   - The tikzpicture options are changed as well: the sibling distance of
  %     level 1 is reduced to 7em and the level distance is increased to 8em.
  %     This is not obligatory but playing around with the sibling and level
  %     distances often helps to make the B+ tree look more beautiful (e.g. the
  %     nodes do not overlap or the orientation of the arrows is better).
  %%%
  \subsection{Insert 5 and 7}
  \begin{center}
  \begin{tikzpicture}[
    BTreeDefault,             % Use default B+ tree style.
    scale = 1,                % No scaling necessary.
    level 1/.style = {        % Level 1 tree style recommended.
      sibling distance = 7em, % Smaller distance betweens siblings.
      level distance = 8em    % Larger distance between levels.
    }
  ] 
    % Set number of pointers per node (this should be done always upfront).
    \SetNoOfPoinersPerNode{4}
  
    % Generate content of (tree) level 0 (root level).
    % Here we have to add a value (5) to the comma-separated list.
    \CreateBTreeNode{0}{1}{{3, 5, 9}}{\levelZeroNodeOne}

    % Generate content of (tree) level 1.
    % We add one new call of the \CreateBTreeNode macro to generate the content
    % one additional leaf node. Because of the splitting, the new node is located
    % between the second and the third existing leaves. Hence, we also have to
    % adjust the node numbers of this level: the previously node 3 now has node
    % number 4 and the new leaf node gets node number 3. Node number 2 keeps its
    % number. Also a new variable is introduced and the variables are used
    % according to the node numbering.
    % Summary:
    %   - Node 1 holds values 1 and 2.
    %   - Node 2 holds values 3 and 4.
    %   - Node 3 holds values 5, 6 and 7.
    %   - Node 4 holds values 9 and 10.
    \CreateBTreeNode{1}{1}{{1, 2}}{\levelOneNodeOne}
    \CreateBTreeNode{1}{2}{{3, 4}}{\levelOneNodeTwo}
    \CreateBTreeNode{1}{3}{{5, 6, 7}}{\levelOneNodeThree}
    \CreateBTreeNode{1}{4}{{9, 10}}{\levelOneNodeFour}

    % Generate B+ tree with one root and four child nodes.
    % Here we add another 'child' clauses generating one new tree node. As before
    % we use the previously generated node contents.
    \node[BTreeNodeDefault] {\levelZeroNodeOne}
    child { node[BTreeNodeDefault] {\levelOneNodeOne} }
    child { node[BTreeNodeDefault] {\levelOneNodeTwo} }
    child { node[BTreeNodeDefault] {\levelOneNodeThree} }
    child { node[BTreeNodeDefault] {\levelOneNodeFour} };

    % Because the root node now has four child nodes, we need to modify the
    % the \ConnectBTreeNodes call accordingly.
    % We connect from level 0 (first argument) node 1 (second argument) to
    % level 1 (third argument). On level 1 we now have 4 nodes to connect (fourth
    % argument) and the node to start with on level 1 has number 1 (fifth
    % argument).
    \ConnectBTreeNodes{0}{1}{1}{4}{1}

    % Because the tree now consists of one more leaf node, we also need to
    % use the \ConnectBTreeLeaves macro accordingly.
    % We connect all leaf nodes of level 1 (first argument) and now there are 4
    % of those to connect.
    \ConnectBTreeLeaves{1}{4}
  \end{tikzpicture}
  \end{center}

  %%%
  % Value 8 is added to the third leaf node. This node is overfilled and is split
  % into two nodes. Afterwards, the leaf level of the tree contains of five nodes
  % but the root node only has four pointers. Thus, an additional level has to
  % added to the tree. The tree afterwards consists of
  %   - one root node holding one value, i.e. {5}.
  %   - two nodes on level 1 holding values {3} and {7, 9}, respectively.
  %   - five leaf nodes on level 2 holding their respective values.
  %
  % This is one of the hardest modifications of an existing B+ tree. The
  % following steps are necessary:
  %   - The root node has to be modified. Because we add another level, the root
  %     node now holds the smallest value of its right subtree, i.e. 5.
  %   - The level of the existing leaf nodes has to be changed to 2 and also
  %     the variable names should be changed accordingly. Although the renaming
  %     is not obligatory as long as the contents are used correctly afterwards,
  %     it is still recommended to keep it tracable and avoid unnecessary errors.
  %   - Two new (inner) nodes have to be generated on level 1. Remember: these
  %     have to be numbered consecutively, i.e. {1}{1} and {1}{2} as the first
  %     two arguments.
  %   - One new leaf nodes has to be generated on level 2, holding the values 7
  %     and 8. This node is generated between the nodes previously numbered 3 and
  %     4. Hence, node number 3 keeps its number, the newly generated node gets
  %     number 4 and the previously content numbered 4 is now node number 5.
  %   - The new level has to be added to the tree environment as well.
  %   - The existing \ConnectBTreeNodes call has to be modified because the root
  %     node now only has two child nodes. Furthermore, two new calls to
  %     \CreateBTreeNodes are necessary to draw the connections between two nodes
  %     of level 1 and the 5 leaf nodes of level 2.
  %   - The \ConnectBTreeLeaves call has to be modified according to the new
  %     level of the leaf nodes and the increased number of leaves.
  %   - The tikzpicture options are changed as well:
  %       - The sibling distance of level 1 is increased to 18em. This is
  %         recommended because of the new level. The sibling distance has to
  %         be increased almost every time when adding a new level.
  %       - The level distance of level 1 is decreased to 6em. This is just a
  %         minor adjustment.
  %       - A new style for level 2 is introduced.
  %       - The whole tikzpicture is scaled down to 0.7 now because the tree got
  %         higher and wider.
  %       - A new style is introduced to scale the nodes too. This is necessary
  %         because nodes have to be scaled in TikZ separately. So, we just use
  %         the BTreeNodeDefault style and add a scale clause for the nodes
  %         afterwards.
  %     This is not obligatory but playing around with the sibling and level
  %     distances often helps to make the B+ tree look more beautiful (e.g. the
  %     nodes do not overlap or the orientation of the arrows is better).
  %%%
  \subsection{Insert 8}
  \begin{center}
  \begin{tikzpicture}[
    BTreeDefault,             % Use default B+ tree style.
    scale = 0.7,              % Scaling to 0.7 necessary.
    level 1/.style = {        % Level 1 tree style recommended.
      sibling distance = 18em,% Larger distance betweens siblings.
      level distance = 6em    % Smaller distance between levels.
    },
    level 2/.style = {        % Level 2 tree style recommended.
      sibling distance = 7em, % Just like the style used for level 1 previously.
      level distance = 6em
    },
    BTreeNodeScaled/.style = {% New B+ tree node style.
      BTreeNodeDefault,       % Uses the predefined B+ tree node style.
      nodes = { scale = 0.7 } % Scales all nodes to 0.7.
    }
  ] 
    % Set number of pointers per node (this should be done always upfront).
    \SetNoOfPoinersPerNode{4}
  
    % Generate content of (tree) level 0 (root level).
    % Because a new level is added, the root has to hold the smallest value of
    % its right subtree, i.e. 5.
    \CreateBTreeNode{0}{1}{{5}}{\levelZeroNodeOne}

    % Generate content of (tree) level 1.
    % Here we generate the new level of the B+ tree consisting of two nodes.
    % Because we push down the leaf level by 1, the new level has level number 1.
    % Numbering is done consecutively, as always, starting with node number 1.
    % The names of the variables are chosen as before to avoid any confusion.
    % Summary:
    %   - Node 1 holds values 3.
    %   - Node 2 holds values 7 and 9.
    \CreateBTreeNode{1}{1}{{3}}{\levelOneNodeOne}
    \CreateBTreeNode{1}{2}{{7, 9}}{\levelOneNodeTwo}

    % Generate content of (tree) level 2.
    % We add one new call of the \CreateBTreeNode macro to generate the content
    % one additional leaf node. Because of the splitting, the new node is located
    % between the third and the fourth existing leaves. Hence, we also have to
    % adjust the node numbers of this level: the previously node 4 now has node
    % number 5 and the new leaf node gets node number 4. Node number 3 keeps its
    % number. Also a new variable is introduced and the variables are used
    % according to the node numbering.
    % Because a new level is added, the leaves are now located at level number 2.
    % This is changed in the first argument for all leaf nodes accordingly.
    % Also the names of the variables are changed accordingly to avoid confusion.
    % Summary:
    %   - Node 1 holds values 1 and 2.
    %   - Node 2 holds values 3 and 4.
    %   - Node 3 holds values 5 and 6.
    %   - Node 4 holds values 7 and 8.
    %   - Node 5 holds values 9 and 10.
    \CreateBTreeNode{2}{1}{{1, 2}}{\levelTwoNodeOne}
    \CreateBTreeNode{2}{2}{{3, 4}}{\levelTwoNodeTwo}
    \CreateBTreeNode{2}{3}{{5, 6}}{\levelTwoNodeThree}
    \CreateBTreeNode{2}{4}{{7, 8}}{\levelTwoNodeFour}
    \CreateBTreeNode{2}{5}{{9, 10}}{\levelTwoNodeFive}

    % Generate B+ tree with one root, two inner nodes and five leaf nodes.
    % Here we add the level according to the new structure of the B+ tree.
    \node[BTreeNodeScaled] {\levelZeroNodeOne}
    child {
      node[BTreeNodeScaled] {\levelOneNodeOne}
      child { node[BTreeNodeScaled] {\levelTwoNodeOne} }
      child { node[BTreeNodeScaled] {\levelTwoNodeTwo} }
    }
    child {
      node[BTreeNodeScaled] {\levelOneNodeTwo}
      child { node[BTreeNodeScaled] {\levelTwoNodeThree} }
      child { node[BTreeNodeScaled] {\levelTwoNodeFour} }
      child { node[BTreeNodeScaled] {\levelTwoNodeFive} }
    };

    % Connect nodes level 0 -> level 1.
    % Because the root now only has two child nodes, we just change the fourth
    % argument accordingly.
    \ConnectBTreeNodes{0}{1}{1}{2}{1}
    % Connect nodes level 1 node 1 -> level 2 nodes 1 and 2.
    % Because we added a new level, we also need to add additional calls to the
    % \ConnectBTreeNodes macro.
    % This call is used to connect the first node of level 1 with its two child
    % nodes of level 2, starting from node number 1 of level 2. So, the nodes
    % 1 and 2 of level 1 are child nodes of node 1 of level 1.
    \ConnectBTreeNodes{1}{1}{2}{2}{1}
    % Connect nodes level 1 node 2 -> level 2 nodes 3, 4 and 5.
    % This call is used to connect the second node of level 1 with its three
    % child nodes of level 2, starting from node number 3 of level 2. So, the
    % nodes 3, 4 and 5 of level 2 are child nodes of node 2 of level 1.
    \ConnectBTreeNodes{1}{2}{2}{3}{3}

    % Because the leaf nodes are now located at level 2 and there is one more
    % leaf node, we need to adjust the \ConnectBTreeLeaves call.
    % We connect all leaf nodes of level 2 (first argument) and now there are 5
    % of those to connect.
    \ConnectBTreeLeaves{2}{5}
  \end{tikzpicture}
  \end{center}

  %%%
  % This section describes all possibilities of shrinking a B+ tree:
  %   - Delete a value in a leaf node, delete corresponding value in inner node
  %     accordingly and connect nodes/leaves
  %   - Change a value in a node/leaf and connect nodes/leaves
  %   - Delete a whole level of the B+ tree and connect nodes/leaves accordingly
  %%%
  \section{Shrinking the B$^+$ tree}

  %%%
  % 
  %%%
  \subsection{Delete 7}
  \begin{center}
  \begin{tikzpicture}[
    BTreeDefault,
    scale = 0.7,
    level 1/.style = {
      sibling distance = 18em,
      level distance = 6em
    },
    level 2/.style = {
      sibling distance = 7em,
      level distance = 6em
    },
    BTreeNodeScaled/.style = {
      BTreeNodeDefault,
      nodes = { scale = 0.7 }
    }
  ] 
    
    \SetNoOfPoinersPerNode{4}
  
    
    \CreateBTreeNode{0}{1}{{5}}{\levelZeroNodeOne}

    
    \CreateBTreeNode{1}{1}{{3}}{\levelOneNodeOne}
    \CreateBTreeNode{1}{2}{{9}}{\levelOneNodeTwo}

    
    \CreateBTreeNode{2}{1}{{1, 2}}{\levelTwoNodeOne}
    \CreateBTreeNode{2}{2}{{3, 4}}{\levelTwoNodeTwo}
    \CreateBTreeNode{2}{3}{{5, 6, 8}}{\levelTwoNodeThree}
    \CreateBTreeNode{2}{4}{{9, 10}}{\levelTwoNodeFour}

    
    \node[BTreeNodeScaled] {\levelZeroNodeOne}
    child {
      node[BTreeNodeScaled] {\levelOneNodeOne}
      child { node[BTreeNodeScaled] {\levelTwoNodeOne} }
      child { node[BTreeNodeScaled] {\levelTwoNodeTwo} }
    }
    child {
      node[BTreeNodeScaled] {\levelOneNodeTwo}
      child { node[BTreeNodeScaled] {\levelTwoNodeThree} }
      child { node[BTreeNodeScaled] {\levelTwoNodeFour} }
    };

    
    \ConnectBTreeNodes{0}{1}{1}{2}{1}
    
    \ConnectBTreeNodes{1}{1}{2}{2}{1}
    
    \ConnectBTreeNodes{1}{2}{2}{2}{3}

    
    \ConnectBTreeLeaves{2}{4}
  \end{tikzpicture}
  \end{center}

  %%%
  % 
  %%%
  \subsection{Delete 10}
  \begin{center}
  \begin{tikzpicture}[
    BTreeDefault,
    scale = 0.7,
    level 1/.style = {
      sibling distance = 18em,
      level distance = 6em
    },
    level 2/.style = {
      sibling distance = 7em,
      level distance = 6em
    },
    BTreeNodeScaled/.style = {
      BTreeNodeDefault,
      nodes = { scale = 0.7 }
    }
  ] 
    
    \SetNoOfPoinersPerNode{4}
  
    
    \CreateBTreeNode{0}{1}{{5}}{\levelZeroNodeOne}

    
    \CreateBTreeNode{1}{1}{{3}}{\levelOneNodeOne}
    \CreateBTreeNode{1}{2}{{8}}{\levelOneNodeTwo}

    
    \CreateBTreeNode{2}{1}{{1, 2}}{\levelTwoNodeOne}
    \CreateBTreeNode{2}{2}{{3, 4}}{\levelTwoNodeTwo}
    \CreateBTreeNode{2}{3}{{5, 6}}{\levelTwoNodeThree}
    \CreateBTreeNode{2}{4}{{8, 9}}{\levelTwoNodeFour}

    
    \node[BTreeNodeScaled] {\levelZeroNodeOne}
    child {
      node[BTreeNodeScaled] {\levelOneNodeOne}
      child { node[BTreeNodeScaled] {\levelTwoNodeOne} }
      child { node[BTreeNodeScaled] {\levelTwoNodeTwo} }
    }
    child {
      node[BTreeNodeScaled] {\levelOneNodeTwo}
      child { node[BTreeNodeScaled] {\levelTwoNodeThree} }
      child { node[BTreeNodeScaled] {\levelTwoNodeFour} }
    };

    
    \ConnectBTreeNodes{0}{1}{1}{2}{1}
    
    \ConnectBTreeNodes{1}{1}{2}{2}{1}
    
    \ConnectBTreeNodes{1}{2}{2}{2}{3}

    
    \ConnectBTreeLeaves{2}{4}
  \end{tikzpicture}
  \end{center}

  %%%
  % 
  %%%
  \subsection{Delete 6}
  \begin{center}
  \begin{tikzpicture}[
    BTreeDefault,
    scale = 1,
    level 1/.style = {
      sibling distance = 8em,
      level distance = 8em
    }
  ] 
    
    \SetNoOfPoinersPerNode{4}
  
    
    \CreateBTreeNode{0}{1}{{3, 5}}{\levelZeroNodeOne}

    
    \CreateBTreeNode{1}{1}{{1, 2}}{\levelOneNodeOne}
    \CreateBTreeNode{1}{2}{{3, 4}}{\levelOneNodeTwo}
    \CreateBTreeNode{1}{3}{{5, 8, 9}}{\levelOneNodeThree}

    
    \node[BTreeNodeDefault] {\levelZeroNodeOne}
    child { node[BTreeNodeDefault] {\levelOneNodeOne} }
    child { node[BTreeNodeDefault] {\levelOneNodeTwo} }
    child { node[BTreeNodeDefault] {\levelOneNodeThree} };

    
    \ConnectBTreeNodes{0}{1}{1}{3}{1}

    
    \ConnectBTreeLeaves{1}{3}
  \end{tikzpicture}
  \end{center}
\end{document}
